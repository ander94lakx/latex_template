% IMPORTANT
% To set an underscore symbol in a URL need to change '_' by '\textunderscore'

%=============================================================================
% PACKAGES
%=============================================================================

\usepackage[spanish]{babel} % For spanish texts (I'm from Spain :))
\usepackage[utf8]{inputenc} % To use accent marks directly
\usepackage{graphicx} % For graphics
\usepackage{cite} % For references
\usepackage{svg} % For vectorial graphics
\usepackage{float} % To modify the pisotion of pictures
\usepackage{listings} % For code
\usepackage{times} % For dates
\usepackage{color} % For colors (only few default colors)
\usepackage[pdftex,
			pdfauthor={My Name},
			pdftitle={PDF Title},
			pdfsubject={Unicorns},
			pdfkeywords={Latex, Unicorns, Vaporwave},
			pdfproducer={Latex},
			pdfcreator={pdflatex}]{hyperref}  % For links and  PDF metadata
\usepackage{url}
\usepackage[Bjornstrup]{fncychap} % For better chapter titles
\usepackage{fancyhdr} % For pages header and footer
\usepackage{titling} % For use doc info in other parts of it 
\usepackage{caption}
\usepackage{minted}

%=============================================================================
% STYLE
%=============================================================================

%-----------------------------------------------------------------------------
% PATH FOR THE PICTURES
%-----------------------------------------------------------------------------

\graphicspath{{./figures/}}

%-----------------------------------------------------------------------------
% FIX FOR THE LIST OF FIGURES
%-----------------------------------------------------------------------------

\makeatletter
\renewcommand*\l@figure{\@dottedtocline{1}{1em}{3.2em}}
\makeatother

%-----------------------------------------------------------------------------
% TYPOGRAPHY
%-----------------------------------------------------------------------------

\renewcommand{\familydefault}{\sfdefault} % Canges typography for ALL document
\setlength{\parskip}{2mm} % Defines the lenght between paragraphs (def == 0)

%-----------------------------------------------------------------------------
% HEADERS AND FOOTERS
%-----------------------------------------------------------------------------
\pagestyle{fancy}
\fancyhf{}
\fancyhead[CE,CO]{\leftmark}
\fancyhead[LE,RO]{\thepage}
\fancyfoot[RE,LO]{My Name}
\fancyfoot[LE,RO]{\thepage}

\renewcommand{\headrulewidth}{2pt}
\renewcommand{\footrulewidth}{1pt}

%-----------------------------------------------------------------------------
% LINKS (All black)
%-----------------------------------------------------------------------------

\hypersetup{
	colorlinks,
	citecolor=black,
	filecolor=black,
	linkcolor=black,
	urlcolor=black
}

%-----------------------------------------------------------------------------
% CONTENTS NUMBERING DEPTH
%-----------------------------------------------------------------------------
\setcounter{tocdepth}{3}
\setcounter{secnumdepth}{3}

%-----------------------------------------------------------------------------
% CODE
%-----------------------------------------------------------------------------

% Color for commands (sh, bash, ...) (with lstlisting)
\definecolor{gray95}{gray}{.95}
\definecolor{gray85}{gray}{.80}
\definecolor{gray45}{gray}{.45}
\definecolor{myturquoise}{RGB}{0, 128, 128}
\definecolor{mypink}{RGB}{177,48,112}
\definecolor{myblue}{RGB}{56,133,231}

\lstset{ 
	frame=Ltb,
	framerule=0pt,
	aboveskip=0.5cm,
	framextopmargin=3pt,
	framexbottommargin=3pt,
	framexleftmargin=0.2cm,
	framesep=0pt,
	rulesep=2.0pt,
	backgroundcolor=\color{gray95},
	rulesepcolor=\color{black},
	%
	stringstyle=\ttfamily,
	showstringspaces = false,
	basicstyle=\small\ttfamily,
	commentstyle=\color{gray45},
	keywordstyle=\bfseries,
	%
	numbers=left,
	numbersep=15pt,
	numberstyle=\tiny,
	numberfirstline = false,
	breaklines=true,
}

% Minimize listings fragmentation
%\lstnewenvironment{listing}[1][]
%{\lstset{#1}\pagebreak[0]}{\pagebreak[0]}

% Code style for terminal
\lstdefinelanguage{none}{}
\lstdefinestyle{terminal}{
	language=none,
	breaklines=true,
	basicstyle=\footnotesize\bf\ttfamily,
	backgroundcolor=\color{gray85},
	rulesepcolor=\color{myturquoise},
	numbers=none,
}

% For the rest of the code (with minted)
% (minted have better syntax highlighting than lstlisting for code)
\usemintedstyle{borland}